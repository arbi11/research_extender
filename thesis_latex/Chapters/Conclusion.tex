\section{Conclusion}

This research aimed to explore the feasibility of modern statistical learning techniques for aiding the design optimization and analysis of electric devices and machines. A number of statistical learning methods, including supervised, bayesian and reinforcement learning have been used in the process. 

The first objective of the thesis was to use modern Deep Learning models to reduce the time spent in analyzing field and performance maps for low-frequency electromagnetic devices as compared to conventional high fidelity solvers based on first principles such as the Finite Element method. The high computation demands of high fidelity FE solutions can limit the effectiveness of these analyses and thus the ability to explore new design possibilities. As an example, for estimating the efficiency map of a single motor drive topology, many operating points need to be simulated using finite element analysis. Thus incorporating the whole efficiency map into the design optimization process is an overwhelmingly time-consuming task and may be impossible, depending on the availability of computational resources. For specific applications in the field of computational electromagnetics, the first section of this thesis (Chapter 2 \& 3) focuses on designing deep learning strategies for computationally inexpensive and accurate predictions of field distribution and performance maps of electric machines. It is demonstrated that DL algorithms can serve as an ideal venue for fast and precise solutions with sufficient training data. In the process, it was observed that this approach faced the issue of non-reliability and extended time spent in generating data. At times, creating massive datasets using Finite Elements simulations can be too expensive and can take weeks and at times months. For handling the enormous requirements of big data, an information transfer strategy between similar problems referred to as Transfer Learning is also implemented. This is verified with application to two test cases - $\eta$ map to $pf$ map prediction of same motor topology and $\eta$ map to $\eta$ map knowledge transfer for different motor topologies. Overall, a reduction of about 40-60\% in overall data requirement was experimentally observed for the two test cases. In scenarios where the NN is not sure about the prediction, a study involving uncertainty analysis in the predictions of DNN is also performed using Bayesian learning. 

The other objective of this thesis involved design optimization, where the excessive time spent in the optimization and manufacturability aspect of the optimized design is addressed. A new TO method (SeqTO) was introduced, which inherently generates manufacturable designs, free from checkerboard patterns, perforations or floating pieces of material and requires significantly less computational resources (around 80\% lower) as compared to using the same optimization algorithm with the ON/OFF discrete method. The SeqTO environment also serves as the platform for exploring tree search-based heuristic algorithms such as Monte Carlo Tree Search and TD-learning, which further reduces the computation burden by about 5-15 \%, compared to using a GA with SeqTO on the same design problem. Leveraging the sequential nature of the controller offered by the SeqTO environment and the pattern learning ability of a NN, a Reinforcement learning agent is also trained, which can generalize to different design parameters, even when such design parameters were unseen during the training phase. It is verified through numerical optimization that the trained agent reduces the number of finite element analysis-based electromagnetic analyses significantly (about 10 times) when deployed for many excitation patterns compared to conventional optimization methods such as the evolutionary algorithms.

Although all the proposed methodologies in this work are capable of providing ``instant" solutions for electromagnetic devices, they can be easily extended to other applications where computer simulations play a prominent role in imitating physical laws across various engineering domains.

Overall the main contributions of the thesis revolve around advancing the usage of high fidelity analysis and designing techniques at an early stage of an electric machine development cycle (V-cycle) as envisioned in Chapter \ref{chapter:1_Intro} (Introduction) of this thesis. It is also aimed to provide practical guidelines and recommendations for successfully deploying modern statistical methods in this field. 

\begin{comment}

Hopefully, this work will persuade more researchers to put efforts in growing this field of research for electromagnetic applications.

This chapter aimed to investigate the applicability of DRL algorithms in the context of TO. Therefore, it is important to explore the different aspects of RL and choose the most suitable DRL methodology. Further, it is also aimed to persuade researchers to put more effort into RL research in the field of TO. Finally, it is hopeful 

  once  the  training  process  is  finished,  the  deep  learning-based RL  method  is  capable  of  yielding  feasible  solutions  without  any  time-consuming  iterative analysis  process.
  
  A significant contribution of this thesis is to advance the usage of high fidelity solution of at anb early stage of the V design cycle, discussed in Chapter \ref{chapter:1_Intro}

Finally, this chapter aimed to investigate the applicability of DRL algorithms in the context of TO. Therefore, it is important to explore the different aspects of RL and choose the most suitable DRL methodology. Further, it is also aimed to persuade researchers to put more effort into RL research in the field of TO.

\end{comment}


\section{Future Work}

A few promising avenues for future work are as follows:

\begin{enumerate}
    \item The work presented in this thesis takes exploratory steps toward using modern statistical learning methodologies in the field. Generalizability is the key for all the techniques. The experiment should be set up such that it can be extended to other similar problems with ease. One such approach for EM analysis can be directly solving the governing partial differential equations (PDEs) using physics-informed deep learning methods. Incorporating the prior knowledge of physical laws in the loss function and training the NN in such a fashion should represent the solution of the PDE. Such an approach is intended to be pursued in the future.

    \item Data collection for training DL networks and FE-based performance evaluation for RL agents is still computationally expensive. 
    \begin{enumerate}
        \item For Supervised Learning tasks, the concept of knowledge transfer is explored in Chapter \ref{chapter:3_RNN}. Only two test cases were subjected to the analysis in this work to test knowledge transfer. Future work that investigates different motor designs and observes the reduction of simulation data needed for training as more knowledge gets accumulated in a multi-task DL network could be fruitful.
        
        \item On the other hand, for RL based problems, it will be interesting to use multiple independent agents which can interact with different instances of the environment in parallel and thus explore an extensive part of the state-action space in much less wall-clock time.
    \end{enumerate}
    \item For TO, only single objective tasks were studied in this work. A multi-physics/multi-objective TO problem can be set up using the techniques used in this work with minor changes to the problem setup. Therefore, studying multi-physics and multi-objective problems will be a possible fruitful research direction. 
    
    \item It is observed that although SeqTO-\textit{v2} significantly increased the generalizability and reduced the computation burden associated with SeqTO-\textit{v1}, it still has limitations in terms of the granularity of the structure that is produced. Therefore, an improved version of the SeqTO environment should be explored in the future, which can balance the requirements of generalizability, computation load, and the necessary fineness of the structure.
    
    \item In this thesis, only two Deep RL algorithms were explored. However, other algorithms and learning methodologies in the literature, such as PPO (Proximal Policy Optimization Algorithms), SAC (Soft Actor-Critic), and curiosity-based learning, are comparatively faster and can be more robust.
\end{enumerate}

